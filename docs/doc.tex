\documentclass{article}

\title{CS3031 - Project 1}
\author{Paolo Moloney - 16325409}

\begin{document}

\maketitle
\newpage

\tableofcontents
\newpage

\section{Specification}

The objective of this project is to implement a proxy server with the following features:

\begin{enumerate}
	\item Respond to HTTP and HTTPS requests, displaying them and the responses on a management console.
	\item Handle websocket connections.
	\item Block selected URLs via the console.
	\item Cache requests locally ro save bandwidth.
	\item Handle multiple requests simultaneously.
\end{enumerate}

\section{Implementation}

I chose to implement the proxy server in Python 3, using the following modules:

\begin{enumerate}
	\item \texttt{socket}: provides low-level access to the BSD socket interface
	\item \texttt{threading}: provides higher-level threading interfaces based on the low-level \texttt{\_thread} module
\end{enumerate}

The following diagram outlines the design decisions I made.

*Diagram*

\section{Code}

\end{document}
